\documentclass{article}
\usepackage[utf8]{inputenc}
\usepackage{changepage}

\begin{document}

\section{Neural Network Algorithm and Optimization Selector}

\begin{itemize}
    \item [\textbf{Esquema del proyecto}]
    \item Trainer
    \begin{itemize}
        \item getInputs:
        \begin {enumerate}
            \item Saca parámetros comunes del nombre del fichero
            \item Recorre Results entrando en las carpetas que tengan los mismos parámetros comunes
            \item Guarda el menor tiempo y algoritmo correspondiente (en un futuro optimizaciones)
            \item Devuelve parámetros comunes + algoritmo (+ optimizaciones)
        \end{enumerate}
        \item getFullInputs:
        \begin {enumerate}
            \item Saca los parámetros de todos los ficheros
        \end{enumerate}
        \item fill:
        \begin {enumerate}
            \item Llama a las dos funciones de getInputs
            \item Escribe los datos en las bases de datos
        \end{enumerate}
        \item train:
        \begin{enumerate}
            \item Llama a fill
            \item Crea un objeto DB que utiliza para hacer fit otra vez con la red neuronal
        \end{enumerate}
    \end{itemize}
    \item Predict
        \begin{itemize}
            \item mutate:
            \begin{enumerate}
                \item Manda los ficheros a Mutomvo (.c y tests) creando la carpeta si hace falta
                \item Ejecuta mutomvo para generar los mutantes del programa
                \item Comprime los mutantes y los manda a la carpeta de la app
                \item Genera los autotests
            \end{enumerate}
            \item predict:
            \begin{enumerate}
                \item Obtiene los inputs mutants, tests, tsSize y lines
                \item Comprueba si hay valor de cores, si no saca el valor del ordenador
                \item Llama a autotest para preparar una ejecución de Malone de solo el programa sin mutar
                \item Realiza la ejecución en Malone y obtiene el valor del tiempo de ejecución
                \item Crea un objeto input con los datos que se pasan por la red
                \item El proceso se realiza dos veces para las optimizaciones 000000 y 100000 y el algoritmo 4
            \end{enumerate}
            \end{itemize}
        \item Autotest
            \begin{itemize}
            \item generate:
            \begin{enumerate}
                \item Genera autotests de todos los posibles modos de ejecución
                \item Genera un fichero bash para ejecutar los autotests
            \end{enumerate}
            \item generateSingle:
            \begin{enumerate}
                \item Igual que generate pero modificando para que solo ejecute el programa original
            \end{enumerate}
            \end{itemize}
    \item Naos
        \begin {itemize}
            \item Interfaz gráfica del proyecto, llama a las funciones:
            \begin{enumerate}
                \item fill
                \item train
                \item mutate
                \item predict
            \end{enumerate}
        \end{itemize}
\end{itemize}

\begin{itemize}
    \item [\textbf{Inputs}]
    \item Nombre del programa
    \item \#Mutantes
    \item \#Tests
    \item \#Lineas del .c
    \item \#Cores
    \item Tiempo total de ejecucion del programa sin mutar
    \item Tamaño del Test Suite
\end{itemize}

\begin{itemize}
    \item Outputs (320)
    \item Algoritmo (1-5)
    \item Optimizaciones (0-1 en cada una de las 6 optimizaciones)
    \item Salida: array de 0s menos en la posición 64*(alg-1)+op a 1
\end{itemize}
\begin{adjustwidth}{-2.5cm}{0cm}
\begin{tabular}{ |c||c|c|c|  }
 \hline
 \multicolumn{4}{|c|}{Apps} \\
 \hline
 Name & line & mutants & description \\
 \hline
 \hline
  add & 13 & 11 & suma dos números \\ \hline
 massive & 13 & 48 & realiza el número pasado como argumento elevado a 5 iteraciones \\ \hline
 factorial & 18 & 30 & calcula el factorial del número pasado \\ \hline
 gcd & 19 & 64 & calcula el mcd de los números pasados \\ \hline
 dictionaryOrder & 29 & 70 & ordena las palabras pasadas alfabéticamente \\ \hline
 primes & 30 & 67 & comprueba si el número pasado es primo \\ \hline
 countWays & 34 & 46 & maneras de subir n escalones de 1 o 2 a la vez \\ \hline
 anagram & 36 & 59 & comprueba si dos cadenas son anagramas entre sí \\ \hline
 transpose & 39 & 148 & traspone la matriz pasada \\ \hline
 linearSearch & 40 & 78 & implementación de linear search \\ \hline
 insertSort & 40 & 107 & implementación de insert sort \\ \hline
 longestPalindrome & 40 & 108 & mayor palíndromo contenido en la palabra \\ \hline
 bubbleSort & 41 & 112 & implementación de bubble sort \\ \hline
 cutRod & 42 & 94 & mayor valor obtenible cortando el array \\ \hline
 selectSort & 45 & 91 & implementación de select sort \\ \hline
 binarySearch & 54 & 126 & implementación de binary search \\ \hline
 quickSort & 55 & 126 & implementación de quick sort \\ \hline
 interpolationSearch & 59 & 180 & implementación de interpolation search \\ \hline
 eggDrop & 60 & 95 & implementación del egg dropping problem \\ \hline
 genPassword & 63 & 48 & genera una contraseña aleatoria \\ \hline
 maxArrSumNeg & 63 & 112 & suma máxima en un array después de K negaciones \\ \hline
 partitionProblem & 66 & 167 & determina si se puede dividir un array en dos subarrays de igual suma \\ \hline
 longestIncrease & 70 & 114 & longest subset in order in the array \\ \hline
 squareMatrix & 71 & 181 & eleva al cuadrado la matriz pasada \\ \hline
 minProd & 72 & 181 & mínimo producto dentro de un array \\ \hline
 exponentialSearch & 80 & 168 & implementación de exponential search \\ \hline
 fibonacciSearch & 86 & 179 & implementación de fibonacci search \\ \hline
 mergeSort & 130 & 218 & algoritmo mergesort \\ \hline
 huffmanCodingEff & 240 & 199 & construye un huffman code tree eficiente \\ \hline
 huffmanCoding & 318 & 314 & construye un huffman code tree \\ \hline
 beaufort & 445 & 481 & cifrado beaufort \\ \hline
 http & 708 & & envía una petición HTTP \\ \hline
 cjson & 1037 & & parsea objetos tipo JSON \\ \hline
 parg & 2518 & 2508 &  \\ \hline
 rnnbit-bit & 4053 & & recurrent neural network for arithmetic \\ \hline
 textgen & 4223 & & text generator \\ \hline
 mnistcnn-cnn & 4328 & & convolutional neural network for mnist database \\ \hline
 ae & 4339 & & tied-weight denoising encoder \\ \hline
 mlp & 4357 & & multi layer perceptron \\ \hline
 vae & 4379 & & variantional autoencoder \\ \hline
 bzip2 & 6998 & & comprime un fichero con el algoritmo bzip2 \\ 
 \hline
\end{tabular}
\end{adjustwidth}
\end{document}
